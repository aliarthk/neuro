\documentclass[11pt]{article}

\usepackage[utf8]{inputenc} % Required for inputting international characters
\usepackage[T1]{fontenc} % Output font encoding for international characters
\usepackage{graphicx}
\usepackage{float}
\usepackage{amsmath}
\usepackage{pdfpages}
\usepackage{mathpazo} % Palatino font
\usepackage[german]{babel}
\parindent0pt
\pdfinclusioncopyfonts=1

\begin{document}

\begin{titlepage} % Suppresses displaying the page number on the title page and the subsequent page counts as page 1
	\newcommand{\HRule}{\rule{\linewidth}{0.5mm}} % Defines a new command for horizontal lines, change thickness here
	
	\center % Centre everything on the page
	\vspace*{0.75cm}
%\includegraphics[width=0.8\textwidth]{../tex/fu_logo}\\[1cm] 

%\textsc{\LARGE  Freie Universität Berlin}\\[1.5cm] % Main heading such as the name of your university/college
	
	\textsc{\Large Neurobiologie für BioinformatikerInnen: Praktikum B}\\[0.65cm] % Major heading such as course name
	
	\textsc{\large Protokoll zum 2. Praktikumstag am 14.01.2019}\\[0.65cm] % Minor heading such as course title

	\HRule\\[0.5cm]
	
	{\huge\bfseries Erregungsleitung im Bauchmark des Regenwurms}\\[0.3cm] % Title of your document
	
	\HRule\\[0.75cm]
	\textsc{\Large\bfseries Gruppe IV}
	\\[0.8cm]
	
\vfill

	\begin{minipage}{0.45\textwidth}
		\begin{flushleft}
			\large
			\textit{Gruppenmitglieder}\\
			\textsc{Alia Rothkegel}\\
			\textsc{Mara Steiger}
			 % Your name
		\end{flushleft}
	\end{minipage}
	~
	\begin{minipage}{0.45\textwidth}
		\begin{flushright}
			\large \vspace{16pt}
			alia.rothkegel@fu-berlin.de\\
			mara.steiger@fu-berlin.de 
		\end{flushright}
	\end{minipage}
	
\vfill

	\begin{minipage}{0.45\textwidth}
		\begin{flushleft}
			\large
			\textit{Lehrveranstalter}\\
			Prof. Dr. P.R. \textsc{Hiesinger}\\ 
			Dr. D. \textsc{Malun}\\ 
			Prof. Dr. M. \textsc{Wernet}
		\end{flushleft}
	\end{minipage}
	~
		\begin{minipage}{0.45\textwidth}
		\begin{flushright}
			
		\end{flushright}
	\end{minipage}
\vfill
	\begin{minipage}{0.7\textwidth}
		\begin{flushleft}
			\large
			\textit{TutorInnen}\\
			\textsc{Lisa Peters}\\
			\textsc{Johannes Brüner Hammacher}\\
			\textsc{Claudia Haushalter}
		\end{flushleft}
	\end{minipage}
	~
		\begin{minipage}{0.2\textwidth}
		\begin{flushright}
			
		\end{flushright}
	\end{minipage}

	% If you don't want a supervisor, uncomment the two lines below and comment the code above
	%{\large\textit{Author}}\\
	%John \textsc{Smith} % Your name
	\vfill\vfill\vfill % Position the date 3/4 down the remaining page

	
	\vfill % Push the date up 1/4 of the remaining page
	
\end{titlepage}

%----------------------------------------------------------------------------------------
\section{Einleitung}
Ziel der heutigen Experimente ist es, die Erregunsleitung und Funktionsweise der dorsalen Riesenaxone des Regenwurmes zu untersuchen. 

\subsection{Erregunsleitung in Nervenzellen}
Die Fortleitung von Informationen wird von Zellen des Nervensystem übernommen und basiert auf Spannungsunterschieden zwischen dessen Zellinnerem und dem extrazellulären Raum. \\
Aufgrund der Semipermeabilität der Membran von Neuronen, liegt ein sogenanntes Ruhepotential von ca. -70mV vor. Die Membran ist für größere Ionen wie Natrium nicht permeabel , aber Kalium kann frei diffundieren. Durch die geringere Konzentration von Kalium-Ionen im Zellinneren, kommt es zu einem Kalium-Austrom d.h. einem Austrom von Kationen, sodass das Nervenzelle gegenüber der Außenseite negativ geladen ist.  


\subsection{Riesenaxone}

\subsection{Myelinisierung}

\subsection{Anatomie des Lumbricus terrestris}

\subsection{Refraktärphase}

\section{Material und Methoden}


\subsection{Versuchsaufbau}


\subsection{Versuchsdurchführung}
\subsubsection{Beobachtung der Lokomotion}

\subsubsection{Identifikation der Riesenfaser bei mechanischer Reizung}

\subsubsection{Bestimmung der Reizschwelle und der Fortleitungsgeschwindigkeit von MRF und LRF durch elektrische Reizung}

\subsubsection{Bestimmung der Refraktärphasen bei elektrischer Reizung}

\section{Ergebnisse}
Durch verschiedene Störfaktoren enthalten unsere Messungen leider viele Artefakte und sind zum Teil unvollständig. 


\section{Diskussion}

\begin{thebibliography}{999}
\bibitem {Quelle1} Autor etc.
\bibitem {Quelle2} Autor etc.
\end{thebibliography}


\end{document}
