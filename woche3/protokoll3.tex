\documentclass[11pt]{article}

\usepackage[utf8]{inputenc} % Required for inputting international characters
\usepackage[T1]{fontenc} % Output font encoding for international characters
\usepackage{graphicx}
\usepackage{float}
\usepackage{hyperref}
\usepackage{amsmath}
\usepackage{cite}
\usepackage{pdfpages}
\usepackage[german]{varioref}
\usepackage{mathpazo} % Palatino font
\usepackage[german]{babel}
\parindent0pt
\pdfinclusioncopyfonts=1

\begin{document}

\begin{titlepage} % Suppresses displaying the page number on the title page and the subsequent page counts as page 1
	\newcommand{\HRule}{\rule{\linewidth}{0.5mm}} % Defines a new command for horizontal lines, change thickness here
	
	\center % Centre everything on the page
	\vspace*{0.75cm}
%\includegraphics[width=0.8\textwidth]{../tex/fu_logo}\\[1cm] 

%\textsc{\LARGE  Freie Universität Berlin}\\[1.5cm] % Main heading such as the name of your university/college
	
	\textsc{\Large Neurobiologie für BioinformatikerInnen: Praktikum B}\\[0.65cm] % Major heading such as course name
	
	\textsc{\large Protokoll zum 3. Praktikumstag am 21.01.2019}\\[0.65cm] % Minor heading such as course title

	\HRule\\[0.5cm]
	
	{\huge\bfseries Neurosim} \\[0.2cm]{\Large Computersimulation von Nervensignalen}\\[0.3cm] % Title of your document
	
	\HRule\\[0.75cm]
	\textsc{\Large\bfseries Gruppe V}
	\\[0.8cm]
	
\vfill

	\begin{minipage}{0.45\textwidth}
		\begin{flushleft}
			\large
			\textit{Gruppenmitglieder}\\
			\textsc{Alia Rothkegel}\\
			\textsc{Mara Steiger}
			 % Your name
		\end{flushleft}
	\end{minipage}
	~
	\begin{minipage}{0.45\textwidth}
		\begin{flushright}
			\large \vspace{16pt}
			alia.rothkegel@fu-berlin.de\\
			mara.steiger@fu-berlin.de 
		\end{flushright}
	\end{minipage}
	
\vfill

	\begin{minipage}{0.45\textwidth}
		\begin{flushleft}
			\large
			\textit{Lehrveranstalter}\\
			Prof. Dr. P.R. \textsc{Hiesinger}\\ 
			Dr. D. \textsc{Malun}\\ 
			Prof. Dr. M. \textsc{Wernet}
		\end{flushleft}
	\end{minipage}
	~
		\begin{minipage}{0.45\textwidth}
		\begin{flushright}
			
		\end{flushright}
	\end{minipage}
\vfill
	\begin{minipage}{0.7\textwidth}
		\begin{flushleft}
			\large
			\textit{TutorInnen}\\
			\textsc{Lisa Peters}\\
			\textsc{Johannes Brüner Hammacher}\\
			\textsc{Claudia Haushalter}
		\end{flushleft}
	\end{minipage}
	~
		\begin{minipage}{0.2\textwidth}
		\begin{flushright}
			
		\end{flushright}
	\end{minipage}

	% If you don't want a supervisor, uncomment the two lines below and comment the code above
	%{\large\textit{Author}}\\
	%John \textsc{Smith} % Your name
	\vfill\vfill\vfill % Position the date 3/4 down the remaining page

	
	\vfill % Push the date up 1/4 of the remaining page
	
\end{titlepage}

%----------------------------------------------------------------------------------------
\section{Einleitung}

\section{Material und Methoden}
\subsection{Material}
Für diesen Versuch benötogten wir lediglich einen Computer mit der Software Neurosim. 

\subsection{Versuchsaufbau}

\subsection{Versuchsdurchführung}

\section{Ergebnisse}
\subsection{Modul Goldmann}

\subsection{Modul Hodgkin - Huxley: Current Clamp}

\begin{table}[H]
\caption{Erhöhen des Stimulus bei 20 Grad C, ausgehend von $40.74\mu A$}
\begin{center}
\begin{tabular}{c|c|c}
Reizstrom in $\mu A$ & Zeit bis zur maximalen Spannung in ms & maximale Spannung in mV \\
\hline\hline
40.74 & 0.84 & 21.82\\
50.74 & 0.61 & 25.09\\
60.74 & 0.52 & 30.55\\
70.74 & 0.44 & 27.27\\
80.74 & 0.42 & 28.36\\
90.74 & 0.4 & 29.45\\
\end{tabular}
\end{center}
\label{werte}
\end{table}


\begin{table}[H]
\caption{Steigern des Reizstroms auf $130 \mu A$}
\begin{center}
\begin{tabular}{c|c|c}
Reizstrom in $\mu A$ & AP pro 100ms & Entladungsfrequenz in Hz \\
\hline\hline
2	&	1	&	10	\\
2,02	&	2	&	20	\\
2,06	&	3	&	30	\\
2,08	&	4	&	40	\\
3	&	5	&	50	\\
5,4	&	6	&	60	\\
10	&	7	&	70	\\
16	&	8	&	80	\\
22	&	9	&	90	\\
30	&	9	&	90	\\
35	&	10	&	100	\\
40	&	10	&	100	\\
45	&	11	&	110	\\
55	&	12	&	120	\\
70	&	12	&	120	\\
75	&	13	&	130	\\
90	&	13	&	130	\\
95	&	14	&	140	\\
115	&	14	&	140	\\
120	&	1	&	10	\\
125	&	1	&	10	\\
130	&	1	&	10	
\end{tabular}
\end{center}
\label{werte}
\end{table}

\subsection{Modul Hodgkin - Huxley: Volatge Clamp}

\begin{table}[H]
\caption{Testen verschiedener Klemmspannungen bei $-100$ mV Haltespannung}
\begin{center}
\begin{tabular}{c|c|c}
Klemmspanung in $-100$ mV & maximaler Stromfluss K$^+$ & maximaler Stromfluss Na$^+$\\
\hline\hline
-50	&	94,41	&	-346,15	\\
-40	&	346,15	&	-1069,93	\\
-30	&	692,31	&	-1793,71	\\
-20	&	1132,87	&	-2230,77	\\
-10	&	1604,9	&	-2409,09	\\
0	&	2108,39	&	-2318,18	\\
10	&	2548,95	&	-2090,91	\\
20	&	2989,51	&	-1608,39	\\
30	&	3398,6	&	-1013,99	\\
40	&	3870,63	&	-363,64	\\
50	&	4248,25	&	342,66	
\end{tabular}
\end{center}
\label{werte}
\end{table}


\subsection{Bewegungswahrnehmung}

\end{document}
